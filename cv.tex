\documentclass{article}
\usepackage[utf8]{inputenc}
\usepackage{hyperref}
\usepackage{multicol}
\usepackage{geometry}
\geometry{hmargin=2.5cm,vmargin=1.5cm}

\title{Curriculum Vitae}
\author{BRUNET GEOFFREY}
\date{}

\begin{document}

\begin{multicols}{2}

\maketitle

\section*{Coordonnées}
10 rue du Grand Caire, Auxerre, 89000 \\
Téléphone: 06.23.35.49.26 \\
Email: geoffrey.brunet@icloud.com \\
Linkedin: \url{https://www.linkedin.com/in/geoffrey-brunet-558315ba/} \\
Github: \url{https://github.com/GeoffreyBrunet}

\section*{Objectif}
En dernière année de bac +5, je souhaite acquérir le poste de software engineer, et me spécialiser dans l’amélioration des performances des applications.

\section*{Expérience}
\begin{itemize}
    \item \textbf{Alternance depuis septembre 2020 chez Quartz Insight en tant que développeur}
    \begin{itemize}
        \item Création d’un outil de gestion de licences pour le logiciel de l’entreprise.
        \item Création d’un outil d’acquisition et de gestion de journaux d’évènements.
        \item Création d’une API REST en java pour gérer et executer des taches planifiées.
    \end{itemize}
    \item \textbf{Alternance de 2 ans dans chez Louis 21 d’aout 2016 à Juillet 2018.}
    \item \textbf{Entreprise Netquarks à Paris en mai/juin 2015 et décembre 2015/février 2016, en stage puis en emploi saisonnier.}
    \item \textbf{Bac professionnel laboratoire contrôle qualité, en 2014, mention ”assez bien”.}
    \item \textbf{Bac professionnel laboratoire contrôle qualité, en 2014, mention ”assez bien”.}
    \item \textbf{Brevet d’études professionnelles travaux de laboratoire, en 2013.}
\end{itemize}

\section*{Diplômes}
\begin{itemize}
    \item Bachelor Concepteur d’applications / Développeur
    \item BTS SIO option Infrastructure Systèmes et Réseaux, en 2018 (major de promotion).
    \item Bac professionnel systèmes électroniques numériques, en 2016, mention ”bien”.
\end{itemize}

\section*{Connaissances Techniques}
\begin{itemize}
    \item \textbf{Langages de programmation:} Typescript, Rust, Java, Lua
    \item \textbf{Éditeurs de texte \& IDEs:} Neovim (configuration maison en lua)
    \item \textbf{Frameworks web:} React, Sveltekit, Angular
    \item \textbf{Librairie CSS:} Tailwind CSS
    \item \textbf{Bases de données:} PostgreSQL, SQLite, Redis
    \item \textbf{Management et gestion des versions:} Git, Docker, Gitlab
    \item \textbf{Web APIs:} Axum, Spring
    \item \textbf{Application multiplateforme:} React Native, Tauri
    \item \textbf{Outils de configurations et build:} NPM, Cargo, Maven
    \item \textbf{Autre:} LaTeX, Markdown
\end{itemize}

\section*{Centres d’Intérêt}
\begin{itemize}
    \item Informatique DIY et impression 3D
    \item Voyages
    \item Running / Trail (Tout terrains)
    \item Photographie argentique et numérique / cinéma
\end{itemize}

\end{multicols}

\end{document}
